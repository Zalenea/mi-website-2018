\chapter{Kommentare zu dargelegten Defiziten oder
Unklarheiten}\label{kommentare-zu-dargelegten-defiziten-oder-unklarheiten}

\section{zu Kriterium 2.1: prägnantere Darstellung der adressierten
Berufsfelder der Studienschwerpunkte im
Master}\label{zu-kriterium-2.1-pruxe4gnantere-darstellung-der-adressierten-berufsfelder-der-studienschwerpunkte-im-master}

\subsection{Auszug aus dem Bericht der
Gutachter}\label{auszug-aus-dem-bericht-der-gutachter}

\begin{siderules}
In den genannten Qualifikationszielen sehen die Gutachter weitestgehend
eine Qualifikation zur Aufnahme einer angemessenen Berufstätigkeit,
merken jedoch an, dass die Beschreibungen der beruflichen Ausrichtung
bei den Studienrichtungen im Masterstudiengang mitunter detaillierter
ausfallen könnten. Sie betonen, dass gerade bei einer derartigen
Aufspaltung in fünf Richtungen für Studieninteressierte deutlich werden
muss, welche Berufsperspektiven mit welcher Studienrichtung verknüpft
werden.
\end{siderules}

Zum besseren Verständnis der Studienschwerpunkte wird die Grundidee der
Schwerpunkte, als auch die damit verbundenen Kompetenzen und
Berufsperspektiven in den einschlägigen Dokumenten (Homepage, Broschüre,
etc.) in nächster Zeit dokumentiert und veröffentlicht. Anbei zwei
exemplarische Berufsbildbeschreibungen für den Schwerpunkt
Human-Computer Interaction:

\textbf{Usability Engineers} arbeiten entweder direkt im Unternehmen
oder in der Beratung von Unternehmen. Ihre maßgebliche Aufgabe ist es,
über den gesamten Lebenszyklus für eine hohe Gebrauchstauglichkeit
interaktiver sozio-technischer Systeme zu sorgen. Dazu wenden sie
Prinzipien, Vorgehensweisen, Methoden und Arbeitstechniken der Disziplin
„Mensch-Computer-Interaktion`` an. Sie planen Entwicklungsprozesse,
analysieren Lebens- und Nutzungskontexte von Nutzergruppen, analysieren
und spezifizieren Nutzungsanforderungen, entwerfen Gestaltungslösungen
und analysieren/evaluieren diese. Darüber hinaus kommunizieren sie mit
allen Berufsgruppen, die bei der Konzeption, Gestaltung, Entwicklung,
Evaluation und dem Betrieb dieser interaktiven Systeme beteiligt sind
und übernehmen damit quasi die Rolle eines Anwalts der Benutzer.

\textbf{Interaction Designer} konzipieren und gestalten die vielfältigen
Beziehungen zwischen Menschen und Technologien. Diese Beziehungen sind
unter anderem ökonomischer, sozialer, ökologischer, kulturell/ethischer
aber auch ästhetischer Art. Anders als bei der eher
ingenieurwissenschaftlichen Herangehensweise der Usability Engineers
denken und handeln Interaction Designer vornehmlich aus der
Designperspektive. Dies bedeutet, dass Interaction Designer in ähnlichen
Projekten tätig sind, aber mit einer ausgeprägten kreativen
Problemlösungskompetenz auf methodischer Ebene sowie einer reflektierten
und eigenverantwortlichen Entscheidungskompetenz ausgestattet sind. Sie
können sicherstellen, dass sich Technologie nach gewünschten
Wertmaßstäben nahtlos und positiv in den Lebensalltag von Menschen
eingliedert. Damit geht Interaction Design weit über die reine
Konzeption und Gestaltung von Eingaben und Ausgabe an der
Benutzungsschnittstelle (User Interface Design) hinaus.

\section{zu Kriterium 2.3: Umfang des Moduls Theoretische
Informatik}\label{zu-kriterium-2.3-umfang-des-moduls-theoretische-informatik}

\subsection{Auszug aus dem Bericht der
Gutachter}\label{auszug-aus-dem-bericht-der-gutachter-1}

\begin{siderules}
Grundsätzlich kommt man darin überein, dass die Theoretische Informatik
gewinnbringend für Studierende sein kann, die Gutachter geben aber zu
bedenken, dass der Umfang von zwei Modulen die Entfaltungsmöglichkeiten
in anderen, der Medieninformatik näheren Themenbereichen, einschränken
kann.
\end{siderules}

\subsection{Stellungnahme der
Hochschule}\label{stellungnahme-der-hochschule}

Die Theoretische Informatik (TI) wird im Medieninformatik Bachelor von
den Programmverantwortlichen als essentiell mit dem jetzigen Umfang von
10 CP angesehen. Ein Ziel bei der Überarbeitung des Studiengangs war es,
formale, algorithmische, mathematische und Realisierungskompetenzen
systematischer und nachhaltiger aufzubauen und im Vergleich zum Status
Quo zu verbessern. TI bereitet dabei ein solides Fundament, welches vor
allem das algorithmische Denken und Abstraktionsvermögen stärkt und
damit die Grundlagen zur Softwaremodellierung legt. An der TH Köln wird
die Theorie in der Theoretischen Informatik auch immer mit konkretem
Praxisbezug vermittelt, damit die Studierenden die Konzepte in späteren
Veranstaltungen wiedererkennen und anwenden können.

\subsection{Beispiele:}\label{beispiele}

\begin{itemize}
\tightlist
\item
  Mengen und Relationen werden durch Bezüge zu Constructive Solid
  Geometry aus der Computergrafik und 3D Druck oder Beziehungen in
  Sozialen Netzwerken dargestellt
\item
  Boolesche Algebra mit Bezug auf Entwurf von Schaltelementen und
  Künstliche Intelligenz zur Lösung logischer Probleme
\item
  Sprachen und Grammatiken, sowie Endliche Automaten, Kellerautomaten,
  Petri-Netze mit konkretem Bezug auf Syntax-Checker,
  Softwaremodellierung und Aufbau von Abstraktionsvermögen durch
  Abbildung alltäglicher Probleme auf eben genannte Darstellungsformen
\end{itemize}

Durch die Turing-Maschinen werden zudem wichtige Informatikkonzepte, wie
das Zerteilen großer Probleme in lösbare Teilprobleme, eine der
wichtigsten Kompetenzen für Informatiker, geübt und auch die Praktische
Umsetzung trainiert, ohne dass man größere Programmiererfahrung
benötigen würde.

Um diesen Praxisbezug in geeigneter Form zu vermitteln ist der
verhältnismäßig hohe Umfang von 10 CP für die TI aus Sicht der
Programmverantwortlichen vollkommen gerechtfertigt. Der Umfang wurde
seitens der Programmverantwortlichen in der Vorbereitung zur
Reakkreditierung jedoch ebenfalls hinlänglich diskutiert. Eine mögliche
Option wäre gewesen, die TI auf 5 CP zu reduzieren. Dann hätten aber
viele Inhalte in anwendungsnähere Module (wie Computergrafik und
Animation, Paradigmen der Programmierung, etc.) verschoben werden
müssen, was den inhaltlichen Umfang der anwendungsnäheren Module
vergrößert hätte. Somit wurde sich darauf verständigt das Modul bei
einem Umfang von 10 CP zu belassen.

\section{\texorpdfstring{zu Kriterium 2.3: Inhaltliche Ausrichtung
des Moduls ``Medienrecht, Medien und
Gesellschaft''}{zu Kriterium 2.3: Inhaltliche Ausrichtung des Moduls Medienrecht, Medien und Gesellschaft}}\label{zu-kriterium-2.3-inhaltliche-ausrichtung-des-moduls-medienrecht-medien-und-gesellschaft}

\subsection{Auszug aus dem Bericht der
Gutachter}\label{auszug-aus-dem-bericht-der-gutachter-2}

\begin{siderules}
Die Gutachter loben die Präsenz dieser beiden Themenbereiche, die in der
Medieninformatik eine immer größere Rolle einnehmen, betonen aber, dass
gerade im Bereich Recht die spezifische inhaltliche Ausrichtung auf
Medienrecht, Internetrecht und Urheberrecht noch stärker betont werden
könnte.
\end{siderules}

\subsection{Stellungnahme der
Hochschule}\label{stellungnahme-der-hochschule-1}

Hier folgen die Programmverantwortlichen der Argumentation der
Gutachter. Bislang handelt es sich bei diesem Modul um ein Modul, dass
für alle Informatik Bachelor Studiengänge am Campus Gummersbach
gemeinsam angeboten wird. Dementsprechend ist die inhaltliche
Ausrichtung der Lehrveranstaltung Medienrecht eher generisch. Die
Programmverantwortlichen streben eine Medieninformatik-spezifische
Lehrveranstaltung an, bei der die oben genannten Themen mehr im Fokus
stehen.

\section{zu Kriterium 2.3: Defizite bei den
Modulbeschreibungen}\label{zu-kriterium-2.3-defizite-bei-den-modulbeschreibungen}

\subsection{Auszug aus dem Bericht der
Gutachter}\label{auszug-aus-dem-bericht-der-gutachter-3}

\begin{siderules}
In Bezug auf die Modulbeschreibungen stellen die Gutachter noch einige
Defizite fest, die im Gespräch mit den Programmverantwortlichen
eingeräumt werden.
\end{siderules}

\subsection{Stellungnahme der
Hochschule}\label{stellungnahme-der-hochschule-2}

Hier folgen die Programmverantwortlichen der Argumentation der
Gutachter. Um in diesem Punkt eine Verbesserung zu erzielen, wird
derzeit ein Leitfaden für die Modulbeschreibungen in der
Medieninformatik entwickelt und in Kürze Anwendung finden.

\section{zu Kriterium 2.3: Verständlichere Darstellung des
Schwerpunktkonzepts im
Master}\label{zu-kriterium-2.3-verstuxe4ndlichere-darstellung-des-schwerpunktkonzepts-im-master}

\subsection{Auszug aus dem Bericht der
Gutachter}\label{auszug-aus-dem-bericht-der-gutachter-4}

\begin{siderules}
Die Gutachter sehen es als notwendig an, hier eine verständlichere
Darstellungsform zu wählen, die den Studierenden das Konzept, die
Strukturierung, die Inhalte und die Anforderungen der Schwerpunkte
zugänglich macht.
\end{siderules}

\subsection{Stellungnahme der
Hochschule}\label{stellungnahme-der-hochschule-3}

Hier folgen die Programmverantwortlichen der Argumentation der
Gutachter. Wie bereits erwähnt, wird zum besseren Verständnis der
Studienschwerpunkte, die Grundidee der Schwerpunkte, als auch die damit
verbundenen Kompetenzen und Berufsperspektiven in den einschlägigen
Dokumenten (Homepage, Broschüre, etc.) in nächster Zeit dokumentiert und
veröffentlicht.

\section{zu Kriterium 2.7: Verhältnis von Studiengangsplätzen und
Studierenden}\label{zu-kriterium-2.7-verhuxe4ltnis-von-studiengangspluxe4tzen-und-studierenden}

\subsection{Auszug aus dem Bericht der
Gutachter}\label{auszug-aus-dem-bericht-der-gutachter-5}

\begin{siderules}
Die Gutachter sind in Anbetracht des großen Engagements der Lehrenden
zwar davon überzeugt, dass alles getan wird, um der großen
Studierendenzahl gerecht zu werden, verweisen aber darauf, dass
langfristig das Verhältnis von Studienplätzen und aufgenommenen
Studierenden wieder angeglichen werden muss. Dies gilt insbesondere mit
Blick auf die Tatsache, dass die Hochschulpaktmittel im Laufe des
Akkreditierungszeitraums auslaufen werden.
\end{siderules}

\subsection{Stellungnahme der
Hochschule}\label{stellungnahme-der-hochschule-4}

Um trotz der aktuellen Überlast angemessene Lehr-Lern Arrangements zu
realisieren, wurden bislang konkrete Maßnahmen ergriffen: - für die
Unterstützung bei Lehrveranstaltungen und Projekten sind eine Reihe von
wissenschaftlichen Mitarbeitern, Lehrbeauftragen und Tutoren eingestellt
worden. Die Finanzierung erfolgt aus Hochschulpaktmitteln und Mitteln
zur ``Verbesserung der Qualität der Lehre'' - Module werden im
Team-Teaching konzipiert und durchgeführt, dabei wird dem
verantwortlichen Dozenten ein Lehrbauftragter zur Seite gestellt, so
dass Lehrveranstaltungen und Workshops parallel durchgeführt werden
können. In diesen Arrangement sind in der Regel auch ein
Wissenschaftlicher Mitarbeiter und ein Tutor beteiligt. - Module nutzen
das Flipped Classroom Konzept, um den Studierenden einerseits den Zugang
zum wissenvermittelnden Material zu erleichtern und die andererseits die
gegebene Kontaktzeit besser zu nutzen - Module eines Semesters werden
sequentiell anstatt parallel durchgeführt. Dabei wird ein Modul in der
ersten Semesterhälfte durchgeführt und das andere in der zweiten
Semesterhälfte, wobei beide Module den Workload des anderen Moduls
nutzen, so dass der Workload für die Studierenden gleich groß bleibt.
Die Modulverantwortlichen haben damit Zugang zu deutliche mehr Räumen
und Ressourcen. Die Studierenden können sich besser auf ein Thema, bzw.
eine Domäne konzentieren, haben also weniger Kontexte gleichzeitig zu
bearbeiten. - stärkere Projektorientierung der anwendungsbezogenen
Module, wobei die Betreuung der Projektteams oftmals von
wissenschaftlichen Mitarbeitern erfolgt. - Zulassungsbeschränkung für
zum Wintersemester 2017/18 - Verlängerung über die Grenze des
Pensionsalters hinaus bei den Professoren Prof.~Dr.~Stenzel und
Prof.~Dr.~Jochum. Die Planstellen der Kollegen sind inzwischen trotzdem
neu besetzt, so dass durch die Überlappung von 2 bis 3 Jahren eine
größere Lehrkapazität zur Verfügung steht. Dieselbe Übergangsregelung
wird auch für die Kollegen/Innen Prof.~Dr.~Faekorn-Woyke,
Prof.~Dr.~Knittel und Prof.~Dr.~Klocke angestrebt.

Eine Übersicht über die verfügbaren Wissenschaftlichen Mitarbeiter und
Lehrbeauftragten\footnote{\href{https://th-koeln.github.io/mi-2017/anhaenge/stellungsnahme/mitarbeiter-und-module-mi-kern-2017.pdf}{Übersicht
  über alle Mitarbeiter und deren Einbindung in die Kernmodule des
  Medieninformatik Bachelor Studiengangs am Campus Gummersbach}}, sowie
deren Finanzierung findet sich im Anhang. Die Finanzierung erfolgt über
Hochschulpakt-Mittel, die zunächst bis zum Ende des Wintersemesters
2018/19 limitiert waren. \textbf{Inzwischen wurden die Mittel bis 2023
verlängt}, was in etwa dem Akkreditierungszeitraum entspricht. Die
wissenschaftlichen Mitarbeiter werden vor allem in projektorientieren
Modulen eingesetzt und übernehmen, neben organisatorischen Aufgaben, vor
allem die Mitbetreuung von Projektgruppen, sowie spezielle Schulungen in
Tools und Arbeitstechniken.

Eine Liste aller Dozenten und deren Beteiligung an Modulen in den
Informatik Studiengängen am Campus
Gummersbach(Lehrverflechtungsmatrix)\footnote{\href{https://th-koeln.github.io/mi-2017/anhaenge/stellungsnahme/dozenten-und-module-2017.pdf}{Übersicht
  über alle Dozenten und deren Module in den Informatik Studiengängen
  des Campus Gummersbach}} finden sich ebenfalls im Anhang.

\section{zu Kriterium 2.8: Fehlende Studien- und
Prüfungsordnungen}\label{zu-kriterium-2.8-fehlende-studien--und-pruxfcfungsordnungen}

\subsection{Auszug aus dem Bericht der
Gutachter}\label{auszug-aus-dem-bericht-der-gutachter-6}

\begin{siderules}
Nach Auskunft der Programmverantwortlichen muss die Studien- und
Prüfungsordnung des Bachelorstudiengangs nicht überarbeitet werden muss,
lediglich der Studienverlaufsplan, der Teil der Prüfungsordnung ist,
muss angepasst werden. Für den Masterstudiengang liegt lediglich der
Entwurf einer Studien- und Prüfungsordnung vor, der noch nicht offiziell
verabschiedet und veröffentlicht wurde. Dies muss für eine abschließende
Akkreditierung nachgeholt werden. Für beide Studiengänge liegen den
Gutachtern Diploma Supplements und Abschlusszeugnisse vor, die sich
jedoch noch auf die älteren Curricula beziehen. Die Gutachter erwarten
auch hierzu die Vorlage der überarbeiteten, angepassten Versionen.
\end{siderules}

\subsection{Stellungnahme der
Hochschule}\label{stellungnahme-der-hochschule-5}

Die Prüfungsordnungen sind inzwischen von den entscheidungstragenden
Gremien verabschiedet worden und müssen lediglich noch veröffentlicht
werden. Überarbeitete Abschlusszeugnisse und Diploma Supplements sind
diesem Dokument angehängt.

\chapter{Nachlieferungen}\label{nachlieferungen}

\section{Die folgenden Nachlieferungen wurden
angefragt}\label{die-folgenden-nachlieferungen-wurden-angefragt}

\subsection{Lehrverflechtungsmatrix}\label{lehrverflechtungsmatrix}

\begin{siderules}
Lehrverflechtungsmatrix für den Studiengang Medieninformatik inklusive
der in der Lehre tätigen Mitarbeiter (zusätzlich zu den Stunden, die in
verwandten Informatikstudiengängen absolviert werden).
\end{siderules}

Die Übersicht über die verfügbaren Wissenschaftlichen Mitarbeiter und
Lehrbeauftragten\footnote{\href{https://th-koeln.github.io/mi-2017/anhaenge/stellungsnahme/mitarbeiter-und-module-mi-kern-2017.pdf}{Übersicht
  über alle Mitarbeiter und deren Einbindung in die Kernmodule des
  Medieninformatik Bachelor Studiengangs am Campus Gummersbach}}, sowie
deren Finanzierung befindet sich im Anhang. Gleiches gilt für die
Übersicht aller Dozenten und deren Beteiligung an Modulen in den
Informatik Studiengängen am Campus
Gummersbach(Lehrverflechtungsmatrix)\footnote{\href{https://th-koeln.github.io/mi-2017/anhaenge/stellungsnahme/dozenten-und-module-2017.pdf}{Übersicht
  über alle Dozenten und deren Module in den Informatik Studiengängen
  des Campus Gummersbach}}.

\subsection{Zeugnismuster}\label{zeugnismuster}

\begin{siderules}
Ggf. Zeugnismuster für die neue Studiengangsstruktur bzw. das neue
Curriculum mit Schwerpunkten. (Ich frage mich, ob nicht auch das DS
geändert werden müsste, um die Schwerpunkte sichtbar zu machen.)
\end{siderules}

Das Zeugnismuster für die neue Studiengangsstruktur befindet sich im
Anhang\footnote{\href{https://th-koeln.github.io/mi-2017/anhaenge/stellungsnahme/Urkunde_Zeugnis_Entwurf_black.pdf}{Zeugnismuster
  für die neue Studiengangsstruktur}}.
